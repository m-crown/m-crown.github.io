%-------------------------
% Adapted from a template by Jake Gutierrez
% Author : Jake Gutierrez
% Based off of: https://github.com/sb2nov/resume
% License : MIT
%------------------------

\documentclass[a4paper,11pt]{article}

\usepackage{latexsym}
\usepackage[empty]{fullpage}
\usepackage{titlesec}
\usepackage{marvosym}
\usepackage[usenames,dvipsnames]{color}
\usepackage{verbatim}
\usepackage{enumitem}
\usepackage[hidelinks]{hyperref}
\usepackage{fancyhdr}
\usepackage[english]{babel}
\usepackage{tabularx}
\input{glyphtounicode}


%----------FONT OPTIONS----------
% sans-serif
% \usepackage[sfdefault]{FiraSans}
% \usepackage[sfdefault]{roboto}
% \usepackage[sfdefault]{noto-sans}
% \usepackage[default]{sourcesanspro}

% serif
% \usepackage{CormorantGaramond}
% \usepackage{charter}


\pagestyle{fancy}
\fancyhf{} % clear all header and footer fields
\fancyfoot{}
\renewcommand{\headrulewidth}{0pt}
\renewcommand{\footrulewidth}{0pt}

% Adjust margins
\addtolength{\oddsidemargin}{-0.5in}
\addtolength{\evensidemargin}{-0.5in}
\addtolength{\textwidth}{1in}
\addtolength{\topmargin}{-.5in}
\addtolength{\textheight}{1.0in}

\urlstyle{same}

\raggedbottom
\raggedright
\setlength{\tabcolsep}{0in}

% Sections formatting
\titleformat{\section}{
  \vspace{-4pt}\scshape\raggedright\large
}{}{0em}{}[\color{black}\titlerule \vspace{-5pt}]

% Ensure that generate pdf is machine readable/ATS parsable
\pdfgentounicode=1

%-------------------------
% Custom commands
\newcommand{\resumeItem}[1]{
  \item\small{
    {#1 \vspace{-2pt}}
  }
}

\newcommand{\resumeSubheading}[4]{
  \vspace{-2pt}\item
    \begin{tabular*}{0.97\textwidth}[t]{l@{\extracolsep{\fill}}r}
      \textbf{#1} & #2 \\
      \textit{\small#3} & \textit{\small #4} \\
    \end{tabular*}\vspace{-7pt}
}

\newcommand{\resumeSubSubheading}[2]{
    \item
    \begin{tabular*}{0.97\textwidth}{l@{\extracolsep{\fill}}r}
      \textit{\small#1} & \textit{\small #2} \\
    \end{tabular*}\vspace{-7pt}
}

\newcommand{\resumeProjectHeading}[2]{
    \item
    \begin{tabular*}{0.97\textwidth}{l@{\extracolsep{\fill}}r}
      \small#1 & #2 \\
    \end{tabular*}\vspace{-7pt}
}

\newcommand{\resumeSubItem}[1]{\resumeItem{#1}\vspace{-4pt}}

\renewcommand\labelitemii{$\vcenter{\hbox{\tiny$\bullet$}}$}

\newcommand{\resumeSubHeadingListStart}{\begin{itemize}[leftmargin=0.15in, label={}]}
\newcommand{\resumeSubHeadingListEnd}{\end{itemize}}
\newcommand{\resumeItemListStart}{\begin{itemize}}
\newcommand{\resumeItemListEnd}{\end{itemize}\vspace{-5pt}}

%-------------------------------------------
%%%%%%  RESUME STARTS HERE  %%%%%%%%%%%%%%%%%%%%%%%%%%%%


\begin{document}

%----------HEADING----------

\begin{center}
    \textbf{\Huge \scshape Matthew Crown} \\ \vspace{4pt}
    \small +43 670 1842390 $|$ \href{mailto:matthewcrown@hotmail.co.uk}{\underline{matthewcrown@hotmail.co.uk}} $|$ 
    \href{https://m-crown.github.io}{\underline{m-crown.github.io}} $|$
    \href{https://linkedin.com/in/matthew-crown}{\underline{linkedin.com/in/matthew-crown}}
\end{center}


%-----------EDUCATION-----------
\section{Ausbildung}
  \resumeSubHeadingListStart
    \resumeSubheading
      {Northumbria University}{Newcastle, UK}
      {PhD in Bioinformatik}{Sep. 2020 -- Oct. 2024}
      \resumeItemListStart
        \resumeItem{Thesis : "Global, viral and protein scale functional annotation tools for -omics and structural bioinformatics.". Weitere Informationen zu Projekten, die während der Promotion entwickelt wurden, finden Sie im Abschnitt „Projekte“.}
      \resumeItemListEnd
    \resumeSubheading
      {Newcastle University}{Newcastle, UK}
      {BSc. (Hons) Biochemistry}{Sep. 2016 -- Jun. 2020}
      \resumeItemListStart
        \resumeItem{Dissertation : “Proteomic analysis of the E3-ubiquitin ligase
        DTX3L interactome”}
        \resumeItem{Abschluss mit Auszeichnung und Auszeichnung mit dem Gus-Lienhard-Preis für Biochemie.}
      \resumeItemListEnd
  \resumeSubHeadingListEnd


%-----------EXPERIENCE-----------
\section{Beschäftigung und Erfahrung}
  \resumeSubHeadingListStart
    \resumeSubheading
      {Leitender wissenschaftlicher Mitarbeiter}{Newcastle, UK}
      {COVID19 Sequencing Group, Northumbria University}{Oct. 2021 -- Sep. 2022}
      \resumeItemListStart
        \resumeItem{Teil der Sequenzierungsbemühungen des COVID19 Genomics Consortium (COG-UK) an der Northumbria University und Entwicklung eines neuartigen strukturellen und funktionellen Annotationstools für SARS-CoV-2-Proteine ​​(siehe Projekte).}
        \resumeItem{Führte im Rahmen der Sequenzierungsbemühungen routinemäßige Analysen der SARS-CoV-2-Sequenzierungsdaten durch, einschliesslich der Ausführung und Fehlerbehebung von Nextflow-Pipelines, der Handhabung sensibler Metadatenverknüpfungen und der Entwicklung von SOPs.}
      \resumeItemListEnd
    \resumeSubheading
      {Industriepraktikum}{Stevenage, UK}
      {Exploratory Biomarker Assay Group, GlaxoSmithKline}{Sep. 2018 -- Aug. 2019}
      \resumeItemListStart
        \resumeItem{1-jähriges Industriepraktikum. Der Schwerpunkt der Arbeit lag auf der Entwicklung und Optimierung von Tests zur Unterstützung präklinischer/klinischer Sicherheits- und Wirksamkeitsstudien von Arzneimitteln mit großen und kleinen Molekülen.}
        \resumeItem{Nutzung kommerzieller Immunoassay- und Durchflusszytometrie-Plattformen und Durchführung einer nachgelagerten Datenanalyse.}
      \resumeItemListEnd
  \resumeSubHeadingListEnd


%-----------PROJECTS-----------
\section{Forschungsprojekte}
    \resumeSubHeadingListStart
      \resumeProjectHeading
        {\textbf{AlphaCognate} $|$ \emph{Python, GEMMI, Nextflow, Git}}{May. 2024 -- Present}
        \resumeItemListStart
          \resumeItem{Entwickelte ein Tool zur Transplantation verwandter Liganden in vorhergesagte Proteinstrukturen, das Annotationsdaten von ProCogGraph (siehe unten) über eine Snakemake-Pipeline integriert und das GEMMI-Paket für die strukturelle Überlagerung vorhergesagter und bekannter Proteinstrukturen verwendet.}
        \resumeItemListEnd
      \resumeProjectHeading
          {\textbf{ProCogGraph} $|$ \emph{neo4j, Python, RDKit, Arpeggio, Nextflow, Git}}{Jan. 2023 -- Jun. 2024}
          \resumeItemListStart
            \resumeItem{Entwickelte eine Graphdatenbank (neo4j) für Enzymdomänen-Ligand-Wechselwirkungen und die Kartierung verwandter Liganden.}
            \resumeItem{Die Datenbank wird mithilfe einer Nextflow-Pipeline erstellt und integriert Chemoinformatik-Ligand-Ähnlichkeitsabgleich und Proteinkontaktanalyse.}
          \resumeItemListEnd
      \resumeProjectHeading
          {\textbf{OMEinfo} $|$ \emph{Docker, Python, Rasterio, Rio-Cogeo, Git}}{Jan. 2023 -- Sep. 2023}
          \resumeItemListStart
            \resumeItem{Entwicklung eines Metadaten-Annotationstools zur automatischen und konsistenten Kommentierung von Standorten mit Geodatenmerkmalen, das es Nicht-Experten ermöglicht, spezielle Geodatenformate (GeoTIFF) in ihre Metadaten zu integrieren.}
          \resumeItemListEnd
      \resumeProjectHeading
          {\textbf{SPEAR} $|$ \emph{Python, Snakemake, minimap2, Plotly, Bash, Git}}{Oct. 2021 -- Sep. 2022}
          \resumeItemListStart
            \resumeItem{Entwickelte eine Pipeline für die schnelle strukturelle und funktionelle Annotation von SARS-CoV-2-Proteinen, einschließlich des Antikörper-Escape-Potenzials und der Konformationsdynamik.}
            \resumeItem{Das Tool umfasst eine Berichtsfunktion, die auch nicht fachkundigen Entscheidungsträgern ein einfaches Verständnis komplexer struktureller/funktionaler Merkmale des Virus ermöglicht.}
          \resumeItemListEnd
    \resumeSubHeadingListEnd
    Eine vollständige Liste der Veröffentlichungen finden Sie in meinem Google Scholar-Profil: \href{https://scholar.google.com/citations?user=b8OKEYcAAAAJ&hl=en&oi=ao}{\underline{Google Scholar}}


%
%-----------PROGRAMMING SKILLS-----------
\section{Technische Fähigkeiten}
 \begin{itemize}[leftmargin=0.15in, label={}]
    \small{\item{
     \textbf{Languages}{: Python, R, Bash, Cypher, HTML/CSS} \\
     \textbf{Reproducible Workflow \& DevOps Tools}{: Snakemake, Nextflow, Git, Docker} \\
     \textbf{Bioinformatics Tools}{: Biopython, GEMMI, RDKit, Pymol, } \\
     \textbf{Generic Tools and Libraries}{: neo4j, pandas, NumPy, Matplotlib, scikit-learn}
     
    }}
 \end{itemize}


%-------------------------------------------
%
%-----------PERSONAL-----------
\section{Persönliche Interessen}
\begin{flushleft}
Außerhalb der Arbeit bin ich ein begeisterter Läufer und habe kürzlich meinen ersten Great North Run für wohltätige Zwecke absolviert. Außerdem genieße ich die körperliche Herausforderung und Problemlösung beim Bouldern. Während der Schwerpunkt meiner Arbeit auf der Bioinformatik liegt, habe ich auch eine Leidenschaft für Technologie und genieße es, die neueste Hardware und Software kennenzulernen und daran herumzubasteln.
\end{flushleft}

%-------------------------------------------
\end{document}